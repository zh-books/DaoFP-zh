\documentclass[DaoFP]{subfiles}
\begin{document}
 \setcounter{chapter}{2}

 \chapter{Isomorphisms\\同构}

 \section{Isomorphic Objects\\同构对象}

 当我们说:
 \[f \circ (g \circ h) = (f \circ g) \circ h \]
 或者:
 \[ f = f \circ id \]
 我们是在断言箭头的\emph{相等性}。左边的箭头是一个操作的结果,而右边的箭头是另一个操作的结果。但结果是\emph{相等}的。

 我们通常通过绘制\emph{交换}图来说明这种相等性,例如:

 \[
  \begin{tikzcd}
   a
   \arrow[r, "h"]
   \arrow[rr, bend left=45, "g \circ h"]
   \arrow[rrr, bend left=80, "f \circ (g \circ h)"]
   \arrow[rrr, bend right=80, "(f \circ g) \circ h"]
   & b
   \arrow[r, "g"]
   \arrow[rr, bend right=45, "f \circ g"]
   &c
   \arrow[r, "f"]
   &d
  \end{tikzcd}
  \begin{tikzcd}
   a
   \arrow[r, "f"]
   \arrow[r, loop, "id"']
   &b
  \end{tikzcd}
 \]

 因此我们比较箭头的相等性。

 我们\emph{不}比较对象的相等性\footnote{半开玩笑地说,在范畴论中调用对象的相等性被认为是“邪恶的”。}。我们将对象视为箭头的汇合点,因此如果我们想比较两个对象,我们会看箭头。

 最简单的两个对象之间的关系是箭头。

 最简单的往返是两个方向相反的箭头的组合。

 \[
  \begin{tikzcd}
   a
   \arrow[r, bend left, "f"]
   & b
   \arrow[l, bend left, "g"]
  \end{tikzcd}
 \]

 可能有两种往返方式。一种是从$a$到$a$的$g \circ f$。另一种是从$b$到$b$的$f \circ g$。

 如果它们两个都生成身份箭头,那么我们说$g$是$f$的\emph{逆元}:
 \[ g \circ f = id_a\]
 \[f \circ g = id_b\]
 我们写作$g = f^{-1}$(读作$f$的\emph{逆元})。箭头$f^{-1}$撤销了箭头$f$的作用。

 这样的一对箭头被称为\emph{同构},这两个对象被称为\emph{同构的}。

 同构的存在告诉我们它连接的两个对象是什么?

 我们说过,对象通过与其他对象的相互作用来描述。因此,让我们考虑从观察者$x$的角度看这两个同构对象的样子。假设有一个箭头$h$从$x$到$a$。

 \[
  \begin{tikzcd}
   & x
   \arrow[ld, red, "h"']
   \\
   a
   \arrow[rr, "f"]
   && b
   \arrow[ll, bend left,  "f^{-1}"]
  \end{tikzcd}
 \]

 有一个对应的箭头从$x$到$b$。它只是$f \circ h$的组合,或者$(f \circ -)$对$h$的作用。

 \[
  \begin{tikzcd}
   & x
   \arrow[ld, "h"']
   \arrow[rd, red, "f \circ h"]
   \\
   a
   \arrow[rr, "f"]
   && b
   \arrow[ll, bend left,  "f^{-1}"]
  \end{tikzcd}
 \]

 类似地,对于任何探测$b$的箭头,有一个对应的探测$a$的箭头。它是$(f^{-1} \circ -)$的作用。

 我们可以使用映射$(f \circ -)$和$(f^{-1} \circ -)$在$a$和$b$之间来回移动焦点。

 我们可以组合这两个映射(见习题\ref{ex-yoneda-composition})以形成一个往返。结果等同于我们应用复合$((f^{-1} \circ f) \circ -)$。但这等于$(id_a \circ  -)$,如我们从习题\ref{ex-yoneda-identity}中所知,它保持箭头不变。

 类似地,由$f \circ f^{-1}$引起的往返保持箭头$x \to b$不变。

 这在两个箭头组之间创建了一个“配对系统”。想象每个箭头向它的配对发送消息,这由$f$或$f^{-1}$决定。然后每个箭头会收到一条消息,而且那将是来自它配对的消息。没有箭头会被遗漏,也没有箭头会收到多于一条消息。数学家称这种配对系统为\emph{双射}或一对一的对应。

 因此,从$x$的角度来看,箭头与箭头地,两个对象$a$和$b$看起来完全相同。就箭头而言,这两个对象之间没有区别。

 两个同构对象具有完全相同的属性。

 特别地,如果你将$x$替换为终端对象$1$,你会看到这两个对象具有相同的元素。对于每个元素$x \colon 1 \to a$,有一个对应的元素$y \colon 1 \to b$,即$y = f \circ x$,反之亦然。同构对象的元素之间存在双射。

 这些不可区分的对象被称为\emph{同构的},因为它们具有“相同的形状”。你看过一个,就等于看过了所有。

 我们将这个同构写作:

 \[a \cong b\]

 当涉及到对象时,我们用同构代替相等。

 在编程中,两个同构的类型具有相同的外部行为。一个类型可以用另一个类型来实现,反之亦然。一个类型可以被另一个类型替换而不改变系统的行为(可能除外性能)。

 在经典逻辑中,如果B从A推导出来,且A从B推导出来,那么A和B在逻辑上是等价的。我们经常说“当且仅当”A为真时,B为真。然而,与逻辑和类型理论之间的前几次平行不同,如果你认为证明是相关的,这次的等价并不那么直截了当。实际上,它导致了一个新的基础数学分支的发展,称为同伦类型论,简称HoTT。

 \begin{exercise}
  提出一个论点,证明从两个同构对象出发的箭头之间存在双射。绘制相应的图。
 \end{exercise}

 \begin{exercise}
  证明每个对象都与自身同构。
 \end{exercise}

 \begin{exercise}
  如果有两个终端对象,证明它们是同构的。
 \end{exercise}

 \begin{exercise}
  证明前一个习题中的同构是唯一的。
 \end{exercise}

 \section{Naturality\\自然性}

 我们已经看到,当两个对象同构时,我们可以使用后组合来在它们之间切换焦点:要么是$(f \circ -)$,要么是$(f^{-1} \circ -)$。

 相反,要在不同的观察者之间切换,我们会使用前组合。

 的确,从$x$到$a$的箭头$h$与从$y$到同一对象的箭头$h\circ g$相关。

 \[
  \begin{tikzcd}
   x
   \arrow[d, "h"']
   && y
   \arrow[ll, dashed, "g"']
   \arrow[dll, red, "h \circ g"']
   \\
   a
   \arrow[rr, "f"]
   && b
   \arrow[ll, bend left,  "f^{-1}"]
  \end{tikzcd}
 \]

 类似地,一个从$x$探测$b$的箭头$h'$对应于一个从$y$探测$b$的箭头$h' \circ g$。

 \[
  \begin{tikzcd}
   x
   \arrow[drr, "h'"]
   && y
   \arrow[ll, dashed, "g"']
   \arrow[d, red, "h' \circ g"]
   \\
   a
   \arrow[rr, "f"]
   && b
   \arrow[ll, bend left,  "f^{-1}"]
  \end{tikzcd}
 \]

 在这两种情况下,我们通过应用前组合$(- \circ g)$来改变视角从$x$到$y$。

 重要的观察是,视角的改变保持了由同构建立的配对系统。如果从$x$的角度来看两个箭头是配对的,那么从$y$的角度来看它们仍然是配对的。这就像说,先用$g$预组合(切换视角)然后再用$f$后组合(切换焦点),或者先用$f$后组合然后再用$g$预组合是无关紧要的。符号上,我们写作:

 \[(- \circ g) \circ (f \circ -) = (f \circ -) \circ (- \circ g)\]

 我们称之为\emph{自然性}条件。

 当你将这个等式应用于一个态射$h \colon x \to a$时,它的意义就会揭示出来。两边都简化为$f \circ h \circ g$。

 \[
  \begin{tikzcd}
   h
   \arrow[r, mapsto, "(- \circ g)"]
   \arrow[d, mapsto, "(f \circ -)"']
   & h \circ g
   \arrow[d, mapsto, "(f \circ -)"]
   \\
   f \circ h
   \arrow[r, mapsto, "(- \circ g)"]
   & f \circ h \circ g
  \end{tikzcd}
 \]

 在这里,自然性条件是由于结合性而自动满足的,但我们很快就会看到它在不太平凡的情况下的推广。

 箭头被用来传播同构的信息。自然性告诉我们,所有的对象都获得了它的一致视图,而不管路径如何。

 我们也可以逆转观察者和被观察者的角色。例如,使用一个箭头$h \colon a \to x$,对象$a$可以从$x$探测任意对象。如果有一个箭头$g \colon x \to y$,它可以将焦点切换到$y$。切换视角到$b$是通过$f^{-1}$的前组合完成的。

 \[
  \begin{tikzcd}
   a
   \arrow[rr, "f"]
   \arrow[d, "h"']
   \arrow[rrd, red, "g\circ h"]
   && b
   \arrow[ll, bend right,  "f^{-1}"']
   \\
   x
   \arrow[rr, dashed, "g"']
   && y
  \end{tikzcd}
 \]

 同样,我们有自然性条件,这次是从同构对的角度来看:

 \[(- \circ f^{-1}) \circ (g \circ -) = (g \circ -) \circ (- \circ f^{-1})\]

 在范畴论中,当我们需要从一个地方移动到另一个地方时,通常需要两个步骤。这里,前组合和后组合的操作可以按任意顺序完成——我们说它们\emph{交换}。但通常情况下,执行步骤的顺序会导致不同的结果。我们经常施加交换条件,并说如果这些条件成立,一个操作与另一个操作是兼容的。

 \begin{exercise}
  证明$f^{-1}$的自然性条件的两边在作用于$h$时简化为:
  \[
   \begin{tikzcd}
    b \arrow[r, "f^{-1}"] &a \arrow[r, "h"] & x \arrow[r, "g"] & y
   \end{tikzcd}
  \]
 \end{exercise}

 \section{Reasoning with Arrows\\使用箭头推理}

 Master Yoneda says: ``At the arrows look!''

 If two objects are isomorphic, they have the same sets of incoming arrows.

 If two objects are isomorphic, they also have the same sets of outgoing arrows.

 If you want to see if two objects are isomorphic, at the arrows look!

 \medskip

 When two objects $a$ and $b$ are isomorphic, any isomorphism $f$ induces a one-to-one mapping $(f \circ -)$ between corresponding sets of arrows.
 \[
  \begin{tikzcd}
   \node(x) at (0, 2) {x};
   \node(a) at (-2, 0) {a};
   \node(b) at (2, 0) {b};
   \node(c1) at (-1, 1.5) {};
   \node(c2) at (-1.5, 1) {};
   \node(c3) at (-1, 2) {};
   \node(c4) at (-2, 1) {};
   \node(d1) at (1, 1.5) {};
   \node(d2) at (1.5, 1) {};
   \node(d3) at (1, 2) {};
   \node(d4) at (2, 1) {};
   \node (aa) at (-1, 0.75) {};
   \node (bb) at (1, 0.75) {};
   \draw[->] (x) .. controls (c1)  and (c2) .. (a); % bend
   \draw[->, green] (x) .. controls (c3)  and (c4) .. (a); % bend
   \draw[->, blue] (x) -- (a);
   \draw[->] (x) .. controls (d1)  and (d2) .. (b); % bend
   \draw[->, green] (x) .. controls (d3)  and (d4) .. (b); % bend
   \draw[->, blue] (x) -- (b);
   \draw[->, red, dashed] (aa) -- node[above]{(f \circ -)} (bb);
   \draw[->] (a) -- node[below]{f} (b);
  \end{tikzcd}
 \]

 函数$(f \circ -)$将每个箭头$h \colon x \to a$映射到一个箭头$f \circ h \colon x \to b$。它的逆$(f^{-1} \circ -)$将每个箭头$h' \colon x \to b$映射到一个箭头$(f^{-1} \circ h')$。

 假设我们不知道这些对象是否同构,但我们知道在每个对象$x$和箭头$h \colon x \to a$之间存在一个可逆映射$\alpha_x$,即在对象$a$和$b$之间的箭头之间存在双射。

 \[
  \begin{tikzcd}
   \node(x) at (0, 2) {x};
   \node(a) at (-2, 0) {a};
   \node(b) at (2, 0) {b};
   \node(c1) at (-1, 1.5) {};
   \node(c2) at (-1.5, 1) {};
   \node(c3) at (-1, 2) {};
   \node(c4) at (-2, 1) {};
   \node(d1) at (1, 1.5) {};
   \node(d2) at (1.5, 1) {};
   \node(d3) at (1, 2) {};
   \node(d4) at (2, 1) {};
   \node (aa) at (-1, 0.75) {};
   \node (bb) at (1, 0.75) {};
   \draw[->] (x) .. controls (c1)  and (c2) .. (a); % bend
   \draw[->, green] (x) .. controls (c3)  and (c4) .. (a); % bend
   \draw[->, blue] (x) -- (a);
   \draw[->] (x) .. controls (d1)  and (d2) .. (b); % bend
   \draw[->, green] (x) .. controls (d3)  and (d4) .. (b); % bend
   \draw[->, blue] (x) -- (b);
   \draw[->, red, dashed] (aa) -- node[above]{\alpha_x} (bb);
  \end{tikzcd}
 \]

 之前,箭头的双射是由同构$f$生成的。现在,箭头的双射是由$\alpha_x$给定的。这是否意味着这两个对象是同构的?我们能从映射$\alpha_x$的族中构造同构$f$吗?答案是肯定的,只要族$\alpha_x$满足自然性条件。

 这是$\alpha_x$对特定箭头$h$的作用。

 \[
  \begin{tikzcd}
   x
   \arrow[d, "h"']
   \arrow[rrd, red, "\alpha_x h"]
   \\
   a
   && b
  \end{tikzcd}
 \]

 这种映射及其逆$\alpha^{-1}_x$将箭头从$x \to b$映射到箭头$x \to a$,如果确实存在同构$f$,它将扮演$(f \circ -)$和$(f^{-1} \circ -)$的角色。映射族$\alpha$描述了一种“人工”的方式来从$a$切换焦点到$b$。

 这是从另一个观察者$y$的角度来看相同的情况:

 \[
  \begin{tikzcd}
   x
   && y
   \arrow[lld, "h'"']
   \arrow[d, red, "\alpha_y h'"]
   \\
   a
   && b
  \end{tikzcd}
 \]

 注意,$y$使用的是同一族中的另一个映射$\alpha_y$。

 只要存在一个箭头$g \colon y \to x$,这两个映射$\alpha_x$和$\alpha_y$就会纠缠在一起。在这种情况下,前组合$(- \circ g)$允许我们切换视角到$y$(注意方向)。

 \[
  \begin{tikzcd}
   x
   \arrow[d, "h"']
   && y
   \arrow[ll, dashed, "g"']
   \arrow[lld, red, "h \circ g"]
   \\
   a
   && b
  \end{tikzcd}
 \]

 我们将焦点切换与视角切换分离。前者由$\alpha$完成,后者由前组合完成。自然性要求这两者之间的兼容性。

 确实,从某个$h$开始,我们可以应用$(- \circ g)$切换到$y$的视角,然后应用$\alpha_y$切换焦点到$b$:
 \[ \alpha_y \circ (- \circ g) \]

 或者我们可以先让$x$使用$\alpha_x$切换焦点到$b$,然后使用$(- \circ g)$切换视角:
 \[ (- \circ g) \circ \alpha_x \]

 在这两种情况下,我们最终都从$y$的角度看到了$b$。我们之前已经做过这个练习,当时我们有一个$a$和$b$之间的同构,我们发现结果是相同的。我们称之为自然性条件。

 如果我们希望$\alpha$给我们一个同构,我们必须强加等效的自然性条件:
 \[ \alpha_y \circ (- \circ g) = (- \circ g) \circ \alpha_x \]

 当作用于箭头$h \colon x \to a$时,我们希望这个图交换:

 \[
  \begin{tikzcd}
   h
   \arrow[r, mapsto, "(- \circ g)"]
   \arrow[d, mapsto, red, "\alpha_x"]
   & h \circ g
   \arrow[d, mapsto, red, "\alpha_y"]
   \\
   \alpha_x h
   \arrow[r, mapsto, "(- \circ g)"]
   &(\alpha_x h) \circ g = \alpha_y (h \circ g)
  \end{tikzcd}
 \]

 这样我们就知道,用$(f \circ -)$替换所有$\alpha$是可行的。但这样的$f$是否存在?我们能从$\alpha$中重建$f$吗?答案是肯定的,我们将使用Yoneda技巧来实现这一点。

 由于$\alpha_x$是为每个对象$x$定义的,它也可以为$a$本身定义。根据定义,$\alpha_a$将一个从$a \to a$的态射映射到$a \to b$的态射。我们确定至少存在一个从$a \to a$的态射,即身份态射$id_a$。事实证明,我们正在寻找的同构$f$由以下公式给出:
 \[f = \alpha_a (id_a)\]

 或者形象地表示为:

 \[
  \begin{tikzcd}
   a
   \arrow[d, "id_a"']
   \arrow[rrd, red, "f = \alpha_a (id_a)"]
   \\
   a
   && b
  \end{tikzcd}
 \]

 让我们验证一下。如果$f$确实是我们的同构,那么对于任意$x$,$\alpha_x$应该等于$(f \circ -)$。为此,我们将自然性条件替换为$a$。我们得到:
 \[\alpha_y(h \circ g) = (\alpha_a h) \circ g \]

 如以下图所示:

 \[
  \begin{tikzcd}
   a
   \arrow[d, "h"']
   \arrow[rrd,  red, "\alpha_a (h)"']
   && y
   \arrow[ll, "g"']
   \arrow[d, red, "\alpha_y (h \circ g)"]
   \\
   a
   && b
  \end{tikzcd}
 \]

 由于$h$的源和目标都是$a$,该等式必须在$h = id_a$时也成立:
 \[\alpha_y (id_a \circ g) = (\alpha_a (id_a)) \circ g \]

 但是$id_a \circ g$等于$g$,而$\alpha_a(id_a)$是我们的$f$,所以我们得到:
 \[\alpha_y g = f \circ g = (f \circ -) g\]

 换句话说,对于每个对象$y$和每个态射$g \colon y \to a$,$\alpha_y = (f \circ -)$。

 注意,尽管$\alpha_x$是为每个$x$和每个从$x \to a$的箭头单独定义的,但事实证明它完全由一个单一的身份箭头的值所决定。这就是自然性的力量!

 \subsection{Reversing the Arrows\\反转箭头}

 正如老子所说,观察者与被观察者之间的二元对立是无法完成的,除非允许观察者与被观察者互换角色。

 再次,我们想证明两个对象$a$和$b$是同构的,但这次我们想将它们视为观察者。箭头$h \colon a \to x$从$a$的角度探测任意对象$x$。之前,当我们知道这两个对象是同构的时,我们可以使用$(- \circ f^{-1})$切换视角到$b$。这次我们有一个变换$\beta_x$。它在箭头$x \to a$和$x \to b$之间建立了双射。

 \[
  \begin{tikzcd}
   x
   \\
   a
   \arrow[u, "h"]
   && b
   \arrow[llu, red, "\beta_x h"']
  \end{tikzcd}
 \]

 如果我们想观察另一个对象$y$,我们将使用$\beta_y$在$a$和$b$之间切换视角,依此类推。

 \[
  \begin{tikzcd}
   \node(x) at (0, 2) {x};
   \node(a) at (-2, 0) {a};
   \node(b) at (2, 0) {b};
   \node(c1) at (-1, 1.5) {};
   \node(c2) at (-1.5, 1) {};
   \node(c3) at (-1, 2) {};
   \node(c4) at (-2, 1) {};
   \node(d1) at (1, 1.5) {};
   \node(d2) at (1.5, 1) {};
   \node(d3) at (1, 2) {};
   \node(d4) at (2, 1) {};
   \node (aa) at (-1, 0.75) {};
   \node (bb) at (1, 0.75) {};
   \draw[<-] (x) .. controls (c1)  and (c2) .. (a); % bend
   \draw[<-, green] (x) .. controls (c3)  and (c4) .. (a); % bend
   \draw[<-, blue] (x) -- (a);
   \draw[<-] (x) .. controls (d1)  and (d2) .. (b); % bend
   \draw[<-, green] (x) .. controls (d3)  and (d4) .. (b); % bend
   \draw[<-, blue] (x) -- (b);
   \draw[->, red, dashed] (aa) -- node[above]{\beta_x} (bb);
  \end{tikzcd}
 \]

 如果两个对象$x$和$y$之间有一个箭头$g \colon x \to y$,那么我们也可以选择使用$(g \circ -)$切换焦点。如果我们想同时切换视角和切换焦点,有两种方法可以做到。自然性要求结果相同:

 \[ (g \circ -) \circ \beta_x = \beta_y \circ (g \circ -) \]

 的确,如果我们用$(- \circ f^{-1})$替换$\beta$,我们将恢复同构的自然性条件。

 \begin{exercise}
  使用身份态射的技巧,从映射族$\beta$中恢复$f^{-1}$。
 \end{exercise}

 \begin{exercise}
  使用前一个习题中的$f^{-1}$,对于任意对象$y$和任意箭头$g \colon a \to y$,求出$\beta_y g$的值。
 \end{exercise}

 正如老子所说:要展示一个同构,定义一个在一万个箭头之间的自然变换往往比找到两个对象之间的一对箭头更容易。

\end{document}

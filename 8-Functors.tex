\documentclass[DaoFP]{subfiles}
\begin{document}
    \setcounter{chapter}{7}

    \chapter{Functors\\函子}

    \section{Categories\\范畴}

    到目前为止,我们只看到了一个范畴——即$types$(类型)和$functions$(函数)的范畴。让我们快速总结一下关于范畴的基本信息。

    范畴是一个$objects$(对象)和在它们之间的$arrows$(箭头)的集合。每一对可组合的箭头都可以进行组合。组合是$associative$(结合律的),并且每个对象都有一个自身环绕的$identity$(单位箭头)。

    类型和函数形成范畴的事实可以通过在$Haskell$中定义组合来表达:

    \begin{haskell}
    (.) :: (b -> c) -> (a -> b) -> (a -> c)
    g . f = \x -> g (f x)
    \end{haskell}

    两个函数$g$和$f$的组合是一个新的函数,它首先将$f$应用于其参数,然后将$g$应用于结果。

    $identity$(单位箭头)是一个多态的“无操作”函数:

    \begin{haskell}
        id :: a -> a
        id x = x
    \end{haskell}

    你可以很容易地验证这种组合是结合律的,并且与$id$组合不会改变函数的行为。

    基于范畴的定义,我们可以想到各种奇怪的范畴。例如,有一个范畴没有对象也没有箭头。它满足所有范畴的条件,只是内容为空。还有一个范畴只包含一个对象和一个箭头(你能猜出那个箭头是什么吗?)。有一个范畴包含两个不相连的对象,还有一个范畴的两个对象通过一个箭头(加上两个单位箭头)连接,等等。这些是我称之为“stick-figure categories”(简易范畴)的例子——只包含少量对象和箭头的范畴。

    \subsection{Category of Sets\\集合的范畴}

    我们还可以将一个范畴剥离掉所有的箭头(除了单位箭头)。这样的纯对象范畴称为$discrete category$(离散范畴)或集合。因为我们将箭头与结构关联,因此集合是一个没有结构的范畴。

    集合形成了它们自己的范畴,称为$\mathbf{Set}$。在这个范畴中的对象是集合,而箭头是集合之间的函数。这些函数被定义为一种特殊的关系,而关系本身被定义为成对的集合。

    在最低的近似下,我们可以在集合的范畴中建模编程。我们通常将类型视为值的集合,将函数视为集合论函数。这没有什么问题。事实上,迄今为止我们描述的所有范畴学构造都有其集合论的根源。范畴积是集合笛卡尔积的推广,和则是并集的推广,等等。

    范畴论提供了更高的精确性:它在绝对必要的结构和多余细节之间进行了细致的区分。

    例如,一个集合论函数不符合我们作为程序员所处理的函数定义。我们的函数必须具有底层算法,因为它们必须能够通过某些物理系统(无论是计算机还是人脑)进行计算。

    \subsection{Opposite Categories\\对偶范畴}

    在编程中,重点是类型和函数的范畴,但我们可以使用这个范畴作为构建其他范畴的起点。

    其中一个范畴称为$opposite category$(对偶范畴)。这是一个所有原始箭头都被反转的范畴:在原始范畴中称为箭头源的部分现在称为目标,反之亦然。

    范畴$\mathcal{C}$的对偶称为$\mathcal{C}^{op}$。当我们讨论$duality$(对偶性)时,我们已经窥见过这个范畴。$\mathcal{C}^{op}$的对象与$\mathcal{C}$的对象相同。

    每当在$\mathcal{C}$中有一个箭头$f \colon a \to b$时,在$\mathcal{C}^{op}$中会有一个对应的箭头$f^{op} \colon b \to a$。

    两个这样的箭头$f^{op} \colon a \to b$和$g^{op} \colon b \to c$的组合$g^{op} \circ f^{op}$由箭头$(f \circ g)^{op}$给出(注意顺序的反转)。

    $\mathcal{C}$中的$terminal object$(终端对象)是$\mathcal{C}^{op}$中的$initial object$(初始对象),$product$(积)在$\mathcal{C}$中对应于$sum$(和)在$\mathcal{C}^{op}$中,等等。

    \subsection{Product Categories\\积范畴}

    给定两个范畴$\mathcal{C}$和$\mathcal{D}$,我们可以构造一个$product category$(积范畴)$\mathcal{C} \times \mathcal{D}$。在这个范畴中的对象是成对的对象$\langle c, d \rangle $,箭头是成对的箭头。

    如果我们在$\mathcal{C}$中有一个箭头$f \colon c \to c'$,并且在$\mathcal{D}$中有一个箭头$g \colon d \to d'$,那么在$\mathcal{C} \times \mathcal{D}$中会有一个对应的箭头$\langle f, g \rangle$。该箭头从$\langle c, d \rangle $指向$\langle c', d' \rangle $,这两者都是$\mathcal{C} \times \mathcal{D}$中的对象。如果它们的组成部分在$\mathcal{C}$和$\mathcal{D}$中分别是可组合的,则可以组合两个这样的箭头。单位箭头是一对单位箭头。

    我们最感兴趣的两个积范畴是$\mathcal{C} \times \mathcal{C}$和$\mathcal{C}^{op} \times \mathcal{C}$,其中$\mathcal{C}$是我们熟悉的类型和函数的范畴。

    在这两个范畴中,对象都是$\mathcal{C}$中的成对对象。在第一个范畴$\mathcal{C} \times \mathcal{C}$中,从$\langle a, b \rangle $到$\langle a', b' \rangle $的$morphism$(态射)是一个对$\langle f \colon a \to a', g \colon b \to b' \rangle $。在第二个范畴$\mathcal{C}^{op} \times \mathcal{C}$中,态射是一个对$\langle f \colon a' \to a, g \colon b \to b' \rangle $,其中第一个箭头方向相反。

    \subsection{Slice Categories\\截面范畴}

    在一个组织良好的宇宙中,对象总是对象,箭头总是箭头。只是有时箭头的集合可以被视为对象。但$slice categories$(截面范畴)打破了这种整洁的分离:它们将单个箭头变成了对象。

    $Slice category$ $\cat C/c$描述了从范畴$\cat C$的角度来看特定对象$c$的方式。它是指向$c$的所有箭头的总和。但要指定一个箭头,我们需要指定它的两个端点。由于其中一个端点固定为$c$,我们只需指定另一个端点。

    在$\mathcal{C}/c$中的一个对象(也称为$over-category$)是一个对$\langle e, p \rangle$,其中$p \colon e \to c$。

    在两个对象$\langle e, p \rangle$和$\langle e', p' \rangle$之间的箭头是$\cat C$的一个箭头$f \colon e \to e'$,它使得以下三角形交换:

    \[
        \begin{tikzcd}
            e
            \arrow[rd, "p"']
            \arrow[rr, "f"]
            && e'
            \arrow[ld, "p'"]
            \\
            &c
        \end{tikzcd}
    \]

    \subsection{Coslice Categories\\余截面范畴}

    存在一个$coslice category$(余截面范畴)$c / \mathcal{C}$的对偶概念,也称为$under-category$。它是从固定对象$c$发出的箭头的范畴。在这个范畴中的对象是成对的$\langle a, i \colon c \to a \rangle$。$c / \mathcal{C}$中的态射是使相关三角形交换的箭头。

    \[
        \begin{tikzcd}
            & c
            \arrow[rd, "j"]
            \arrow[ld, "i"']
            \\
            a
            \arrow[rr, "f"']
            && b
        \end{tikzcd}
    \]

    特别地,如果范畴$\mathcal{C}$有一个$terminal object$(终端对象)$1$,那么$coslice$ $1 / \mathcal{C}$中的对象就是$\mathcal{C}$中所有对象的$global elements$(全局元素)。

    $1/  \mathcal{C}$的态射对应于箭头$f \colon a \to b$,它将$a$的全局元素集合映射到$b$的全局元素集合。

    \[
        \begin{tikzcd}
            & 1
            \arrow[rd, "y"]
            \arrow[ld, "x"']
            \\
            a
            \arrow[rr, "f"']
            && b
        \end{tikzcd}
    \]
    特别地,从类型和函数的范畴构建余截面范畴验证了我们将类型视为值的集合的直觉,其中值由类型的全局元素表示。

    \section{Functors\\函子}

    我们在讨论代数数据类型时已经看到了$functoriality$(函子性)的例子。其思想是这种数据类型“记住”了其创建的方式,并且我们可以通过将一个箭头应用到其“内容”上来操控这种记忆。

    在某些情况下,这种直觉非常有说服力:我们将$product type$(积类型)视为一个包含其成分的对。毕竟,我们可以通过$projections$(投影)来获取它们。

    在$function objects$(函数对象)的情况下,这一点不太明显。你可以将函数对象可视化为暗地里存储所有可能的结果,并使用函数参数对它们进行索引。一个从$Bool$到其他类型的函数显然等同于一个值对,一个用于$True$,另一个用于$False$。将某些函数实现为$lookup tables$(查找表)是一个众所周知的编程技巧。它称为$memoization$(备忘录化)。

    尽管将以自然数作为参数的函数备忘录化并不现实;我们仍然可以将它们概念化为(无限或甚至不可数的)查找表。

    如果你可以将数据类型视为值的容器,那么将函数应用于所有这些值并创建一个变换后的容器是有意义的。当这种可能存在时,我们称该数据类型是$functorial$(函子性的)。

    同样,函数类型需要更多的“怀疑暂停”。你将函数对象视为一个查找表,以某种类型为键。如果你想使用另一种相关类型作为键,你需要一个将新键转换为原始键的函数。这就是为什么函数对象的函子性之一是反向箭头:

    \begin{haskell}
        dimap :: (a' -> a) -> (b -> b') -> (a -> b) -> (a' -> b')
        dimap f g h = g . h . f
    \end{haskell}

    你正在将转换应用于一个具有对类型$a$的值作出响应的“受体”的函数$h :: a -> b$,并且你希望使用它来处理$a'$类型的输入。这仅在你有一个从$a'$到$a$的转换器时可能,即$f :: a' -> a$。

    一个数据类型“包含”另一个类型的值的概念也可以通过说一个数据类型被另一个数据类型参数化来表达。例如,类型$List$ $a$是由类型$a$参数化的。

    换句话说,$List$将类型$a$映射为类型$List a$。没有参数的$List$本身被称为$type constructor$(类型构造子)。

    \subsection{Functors between Categories\\范畴之间的函子}

    在范畴论中,类型构造子被建模为对象到对象的映射。它是一个作用于对象的函数。这不要与对象之间的箭头混淆,后者是范畴结构的一部分。

    事实上,想象一个$categories$(范畴之间)的映射更容易。源范畴中的每个对象都映射到目标范畴中的一个对象。如果$a$是$\mathcal{C}$中的一个对象,那么在$\mathcal{D}$中有一个对应的对象$F a$。

    $Functorial mapping$(函子映射),或$functor$(函子),不仅映射对象,还映射它们之间的箭头。在第一个范畴中的每个箭头
    $f \colon a \to b$在第二个范畴中都有一个对应的箭头:
    $F f \colon F a \to F b$。

    \[
        \begin{tikzcd}
            a
            \arrow[d, blue, "f"]
            \arrow[rr, dashed]
            && F a
            \arrow[d, red, "F f"]
            \\
            b
            \arrow[rr, dashed]
            && F b
        \end{tikzcd}
    \]

    我们使用相同的字母(此处为$F$)来命名对象映射和箭头映射。

    如果范畴提炼出$structure$(结构)的本质,那么函子就是保留这种结构的映射。在源范畴中相关的对象在目标范畴中也相关。

    范畴的结构由箭头及其组合定义。因此,函子必须保留组合。在一个范畴中组合的内容:
    $h = g \circ f$
    应该在第二个范畴中保持组合:
    $F h = F (g \circ f) = F g \circ F f$

    我们可以在$\mathcal{C}$中组合两个箭头并将复合映射到$\mathcal{D}$,或者我们可以映射单个箭头,然后在$\mathcal{D}$中组合它们。我们要求结果相同。

    \[
        \begin{tikzcd}
            a
            \arrow[d, "f"]
            \arrow[rrr, dashed]
            \arrow[dd, bend right = 70, blue, "g \circ f"']
            &&& F a
            \arrow[d, "F f"]
            \arrow[dd, bend left = 70, "F g \circ F f"]
            \arrow[dd, bend right = 70, red, "F (g \circ f)"']
            \\
            b
            \arrow[d, "g"]
            &&& F b
            \arrow[d, "F g"]
            \\
            c
            \arrow[rrr, dashed]
            &&& F c
        \end{tikzcd}
    \]

    最后,函子必须保留$identity arrows$(单位箭头):
    $F\, id_a = id_{F a}$

    \[
        \begin{tikzcd}
            a
            \arrow[loop, blue,  "id_a"']
            \arrow[rr, dashed]
            && F a
            \arrow[loop, red, "F \, id_{a}"']
            \arrow[loop, controls={+(2.5, 2.6) and +(-2, 2.5)}, "id_{F a}"']
        \end{tikzcd}
    \]

    这些条件一起定义了什么是保持范畴结构的函子。

    同样重要的是要理解什么条件不属于定义的一部分。例如,允许一个函子将多个对象映射为一个对象。它还可以将多个箭头映射为一个箭头,只要它们的端点匹配。

    极端情况下,任何范畴都可以映射到一个只有一个对象和一个箭头的单例范畴。

    此外,目标范畴中的所有对象或箭头不一定都必须由函子覆盖。极端情况下,我们可以将一个单例范畴的函子映射到任何(非空)范畴。这样的函子选择一个单一对象及其单位箭头。

    $Constant functor$(常量函子)$\Delta_c$是一个函子的例子,它将源范畴中的所有对象映射到目标范畴中的一个对象$c$,并将源范畴中的所有箭头映射到该对象的单位箭头$id_c$。

    在范畴论中,函子常用于在另一个范畴内部创建一个范畴的模型。它们能够将多个对象和箭头合并为一个,这意味着它们生成了源范畴的简化视图。它们“抽象”了源范畴的某些方面。

    它们可能只覆盖目标范畴的某些部分,这意味着这些模型嵌入在一个更大的环境中。

    从某些极小的“stick-figure categories”简易范畴(例如具有两个对象和一个箭头的范畴)中导出的函子可用于定义更大范畴中的模式。

    \begin{exercise}
        描述一个源自“$walking arrow$”范畴的函子。它是一个只有两个对象和一个箭头的简易范畴(再加上必需的单位箭头)。
        \[
            \begin{tikzcd}
                a
                \arrow[loop,  "id_{a}"']
                \arrow[r, "f"]
                & b
                \arrow[loop, "id_{b}"']
            \end{tikzcd}
        \]
    \end{exercise}

    \begin{exercise}
        “$walking iso$”范畴与“$walking arrow$”范畴相似,但多了一个从$b$回到$a$的箭头。证明从这个范畴出发的函子总是选择目标范畴中的一个同构。
    \end{exercise}

    \section{Functors in Programming\\编程中的函子}

    $Endofunctors$(自函子)是最容易在编程语言中表达的函子类。这些函子将一个范畴(此处为类型和函数的范畴)映射回自身。

    \subsection{Endofunctors\\自函子}

    自函子的第一部分是将类型映射到类型。这是通过类型构造子完成的,它们是类型级别的函数。

    $List$类型构造子将一个任意类型$a$映射为类型$List a$。

    $Maybe$类型构造子将$a$映射为$Maybe a$。

    自函子的第二部分是箭头的映射。给定一个函数$a \to b$,我们希望能够定义一个函数$List a \to List b$或$Maybe a \to Maybe b$。这就是我们之前讨论过的数据类型的$functoriality$(函子性)。函子性使我们能够$lift$(提升)一个任意的函数到变换后的类型之间的函数。

    函子性可以在$Haskell$中通过一个$typeclass$(类型类)来表达。在这种情况下,类型类由类型构造子$f$参数化(在$Haskell$中我们使用小写名称作为类型构造子变量)。我们说$f$是一个$Functor$,如果存在一个相应的函数映射,称为$fmap$:

    \begin{haskell}
        class Functor f where
        fmap :: (a -> b) -> (f a -> f b)
    \end{haskell}

    编译器知道$f$是一个类型构造子,因为它被应用于类型,例如$f a$和$f b$。

    要向编译器证明特定类型构造子是一个$Functor$,我们必须为其提供$fmap$的实现。这通过定义类型类$Functor$的一个$instance$(实例)来实现。例如:

    \begin{haskell}
        instance Functor Maybe where
        fmap g Nothing  = Nothing
        fmap g (Just a) = Just (g a)
    \end{haskell}

    一个函子还必须满足一些定律:它必须保持组合性和单位性。这些定律不能在$Haskell$中表达,但应该由程序员检查。我们之前已经看到过一个$badMap$的定义,它不满足单位定律,但它仍然会被编译器接受。它为列表类型构造子$[]$定义了一个“非法”的$Functor$实例。

    \begin{exercise}
        证明$WithInt$是一个函子:
        \begin{haskell}
            data WithInt a = WithInt a Int
        \end{haskell}
    \end{exercise}

    有一些基本的函子可能看起来很简单,但它们作为其他函子的构建块发挥了作用。

    我们有一个$identity endofunctor$(恒等自函子),它将所有对象映射到自身,并将所有箭头映射到自身。

    \begin{haskell}
        data Id a = Id a
    \end{haskell}

    \begin{exercise}
        证明$Id$是一个$Functor$。提示:为其实现$Functor$实例。
    \end{exercise}

    我们还有一个常量函子$\Delta_c$,它将所有对象映射到一个单一对象$c$,并将所有箭头映射到该对象的单位箭头$id_c$。在$Haskell$中,它是一个以目标对象$c$为参数的函子族:

    \begin{haskell}
        data Const c a = Const c
    \end{haskell}

    这个类型构造子忽略了其第二个参数。

    \begin{exercise}
        证明$(Const c)$是一个$Functor$。提示:该类型构造子接受两个参数,但在$Functor$实例中,它只对第一个参数部分应用。在第二个参数中,它是函子性的。
    \end{exercise}

    \subsection{Bifunctors\\双函子}

    我们还看到了一些类型构造子,它们接受两个类型作为参数:如积和和类型。它们也是函子性的,但它们提升一对函数而不是一个函数。在范畴论中,我们将这些定义为从积范畴$\mathcal{C} \times \mathcal{C}$到$\mathcal{C}$的函子。

    这样的函子将一对对象映射到一个对象,并将一对箭头映射到一个箭头。

    在$Haskell$中,我们将这样的函子视为$Bifunctor$类型类的成员。

    \begin{haskell}
        class Bifunctor f where
        bimap :: (a -> a') -> (b -> b') -> (f a b -> f a' b')
    \end{haskell}

    编译器通过查看它被应用于两个类型(例如$f a b$)来推断$f$是一个两个参数的类型构造子。

    要向编译器证明特定类型构造子是一个$Bifunctor$,我们需要定义一个实例。例如,可以这样定义一个对的双函子性:

    \begin{haskell}
        instance Bifunctor (,) where
        bimap g h (a, b) = (g a, h b)
    \end{haskell}

    \begin{exercise}
        证明$MoreThanA$是一个双函子:
        \begin{haskell}
            data MoreThanA a b = More a (Maybe b)
        \end{haskell}
    \end{exercise}

    \subsection{Contravariant Functors\\逆变函子}

    从对偶范畴$\mathcal{C}^{op}$出发的函子称为$contravariant$(逆变)的。它们的特点是提升箭头的方向相反。通常的函子有时称为$covariant$(协变)的。

    在$Haskell$中,逆变函子构成了一个类型类$Contravariant$:

    \begin{haskell}
        class Contravariant f where
        contramap :: (b -> a) -> (f a -> f b)
    \end{haskell}

    将函子分为生产者和消费者有时很方便。在这个类比中,协变函子是一个生产者。你可以通过应用$fmap$一个$a -> b$的函数,将一个$a$的生产者转变为一个$b$的生产者。相反,要将一个$a$的消费者转变为一个$b$的消费者,你需要一个反向的函数$b -> a$。

    例子:一个谓词是一个返回$True$或$False$的函数:

    \begin{haskell}
        data Predicate a = Predicate (a -> Bool)
    \end{haskell}

    很容易看出它是一个逆变函子:

    \begin{haskell}
        instance Contravariant Predicate where
        contramap f (Predicate h) = Predicate (h . f)
    \end{haskell}

    实际上,逆变函子的唯一非平凡例子是函数对象的各种变体。

    一种判断给定函数类型在其类型参数上是协变还是逆变的方法是通过分配极性。我们说函数的返回类型处于正位置,因此它是协变的;参数类型处于负位置,因此它是逆变的。但是,如果你将整个函数对象放在另一个函数的负位置上,那么它的极性就会反转。

    考虑这个数据类型:

    \begin{haskell}
        data Tester a = Tester ((a -> Bool) -> Bool)
    \end{haskell}

    它将$a$放在双负位置上,因此是正位置。因此它是一个协变的$Functor$。它作为$a$的生产者:

    \begin{haskell}
        instance Functor Tester where
        fmap f (Tester g) = Tester g'
        where g' h = g (h . f)
    \end{haskell}

    注意这里的括号是重要的。类似的函数$a -> Bool -> Bool$将$a$放在负位置上。因为它是一个从$a$返回函数$(Bool -> Bool)$的函数。等价地,你可以将其变为一个接受对的函数:$(a, Bool) -> Bool$。无论哪种方式,$a$最终都在负位置上。

    \subsection{Profunctors\\变截函子}

    我们之前看到函数类型是函子性的。它一次提升两个函数,就像$Bifunctor$一样,只不过其中一个函数的方向相反。

    在范畴论中,这对应于一个从$\mathcal{C}^{op} \times \mathcal{C}$到$\mathbf{Set}$的函子。这样的函子称为$Profunctors$(变截函子)。

    在$Haskell$中,变截函子形成了一个类型类$Profunctor$:

    \begin{haskell}
        class Profunctor f where
        dimap :: (a' -> a) -> (b -> b') -> (f a b -> f a' b')
    \end{haskell}

    你可以将一个变截函子视为一个同时是生产者和消费者的类型。它消耗一种类型并产生另一种类型。

    函数类型$a -> b$可以写成中缀运算符$->$,它是$Profunctor$的一个实例:

    \begin{haskell}
        instance Profunctor (->) where
        dimap f g h = g . h . f
    \end{haskell}

    这与我们的直觉一致,即一个函数$a -> b$消耗类型$a$的参数并产生类型$b$的结果。

    在编程中,所有非平凡的变截函子都是函数类型的变体。

    \section{The Hom-Functor\\Hom函子}

    两个对象之间的箭头形成一个集合。这个集合称为$hom-set$(同态集),通常写作$\mathcal{C}(a, b)$。

    我们可以将$hom-set$ $\mathcal{C}(a, b)$解释为从$a$观察$b$的所有方式。

    另一种看待$hom-sets$的方式是将它们视为一个映射,它将一个集合$\mathcal{C}(a, b)$分配给每对对象。集合本身是$\mathbf{Set}$范畴中的对象。所以我们有一个范畴之间的映射。

    这个映射是函子性的。要看到这一点,让我们考虑当我们转换两个对象$a$和$b$时会发生什么。我们感兴趣的是一个将集合$\mathcal{C}(a, b)$映射到集合$\mathcal{C}(a', b')$的转换。$\mathbf{Set}$中的箭头是常规函数,因此只需定义它们在集合上对单个元素的操作。

    $\mathcal{C}(a, b)$的元素是一个箭头$h \colon a \to b$,$\mathcal{C}(a', b')$的元素是一个箭头$h' \colon a' \to b'$。我们知道如何将一个转换为另一个:我们需要用箭头$g' \colon a' \to a$对$h$进行前组合,并用箭头$g \colon b \to b'$对它进行后组合。

    换句话说,将对$\langle a, b \rangle$的一对对象映射为集合$\mathcal{C}(a, b)$的映射是一个$profunctor$:

    $\mathcal{C}^{op} \times \mathcal{C} \to \mathbf{Set}$

    我们经常感兴趣的是在保持一个对象固定的情况下变换另一个对象。当我们固定源对象并变换目标对象时,结果是一个函子,写作:

    $\mathcal{C}(a, -) \colon \mathcal{C} \to \mathbf{Set}$

    这个函子在箭头$g \colon b \to b'$上的作用写作:

    $\mathcal{C}(a, g) \colon \mathcal{C}(a, b) \to \mathcal{C}(a, b')$

    它通过后组合给出:

    $\mathcal{C}(a, g) = (g \circ -)$

    变换$b$意味着从一个对象切换到另一个对象,因此完整的函子$\mathcal{C}(a, -)$将所有从$a$发出的箭头组合成一个一致的范畴视图。这是“a的世界观”。

    相反,当我们固定目标对象并变换源对象时,我们得到一个逆变函子:

    $\mathcal{C}(-, b) \colon \mathcal{C}^{op} \to \mathbf{Set}$

    其作用在箭头$g' \colon a' \to a$上写作:

    $\mathcal{C}(g', b) \colon \mathcal{C}(a, b) \to \mathcal{C}(a', b)$

    它通过前组合给出:

    $\mathcal{C}(g', b) = (- \circ g')$

    函子$\mathcal{C}(-, b)$将所有指向$b$的箭头组织成一个一致的视图。它是“b的世界观”。

    我们现在可以重新表述同构章节中的结果。如果两个对象$a$和$b$是同构的,那么它们的$hom-sets$也是同构的。特别是:

    $\mathcal{C}(a, x) \cong \mathcal{C}(b, x)$

    和

    $\mathcal{C}(x, a) \cong \mathcal{C}(x, b)$

    我们将在下一章讨论自然性条件。

    另一种看待$hom-functor$ $\cat C(a, -)$的方法是将其视为一个提供“a是否与我相连?”问题答案的$oracle$(神谕)。如果集合$\cat C(a, x)$为空,答案是否定的:“a与x不相连。”否则,集合$\cat C(a, x)$中的每个元素都是存在这种连接的证明。

    相反,逆变函子$\cat C (-, a)$回答的问题是:“我与a相连吗?”

    综上所述,$profunctor$ $\cat C(x, y)$在对象之间建立了一个$proof-relevant relation$(证明相关关系)。集合$\cat C(x, y)$中的每个元素都是$x$与$y$相连的证明。如果集合为空,这两个对象是无关的。

    \section{Functor Composition\\函子的组合}

    就像我们可以组合函数一样,我们也可以组合函子。两个函子是可组合的,如果一个函子的目标范畴与另一个的源范畴相同。

    在对象上,$functor composition$(函子组合)$G$之后的$F$首先将$F$应用于一个对象,然后将$G$应用于结果;在箭头上也是如此。

    显然,你只能组合可组合的函子。然而,所有$endofunctors$(自函子)都是可组合的,因为它们的目标范畴与源范畴相同。

    在$Haskell$中,函子是一个参数化的数据类型,因此两个函子的组合也是一个参数化的数据类型。在对象上,我们定义:

    \begin{haskell}
        data Compose g f a = Compose (g (f a))
    \end{haskell}

    编译器推断$f$和$g$必须是类型构造子,因为它们被应用于类型:$f$被应用于类型参数$a$,$g$被应用于结果类型。

    或者,你可以告诉编译器前两个参数是类型构造子。你可以通过提供一个$kind signature$(种类签名)来实现这一点,这需要一个语言扩展$KindSignatures$,你可以在源文件的顶部添加:

    \begin{haskell}
    {-# language KindSignatures #-}
    \end{haskell}

    你还应该导入$Data.Kind$库,该库定义了$Type$:

    \begin{haskell}
        import Data.Kind
    \end{haskell}

    种类签名就像类型签名一样,只不过它可以用于描述操作类型的函数。

    常规类型具有种类$Type$。类型构造子具有种类$Type -> Type$,因为它们将类型映射为类型。

    $Compose$接受两个类型构造子并生成一个类型构造子,因此它的种类签名是:

    \begin{haskell}
    (Type -> Type) -> (Type -> Type) -> (Type -> Type)
    \end{haskell}

    完整的定义是:

    \begin{haskell}
        data Compose :: (Type -> Type) -> (Type -> Type) -> (Type -> Type)
        where
        Compose :: (g (f a)) -> Compose g f a
    \end{haskell}

    任何两个类型构造子都可以这样组合。目前没有要求它们必须是函子。

    但是,如果我们想提升一个函数使用类型构造子的组合$g$之后的$f$,那么它们必须是函子。此要求作为约束在实例声明中编码:

    \begin{haskell}
        instance (Functor g, Functor f) => Functor (Compose g f) where
        fmap h (Compose gfa) = Compose (fmap (fmap h) gfa)
    \end{haskell}

    约束$ (Functor g, Functor f) $ 表示条件是两个类型构造子必须是$Functor$类的实例。约束后跟随双箭头。

    我们正在建立函子性的类型构造子是$Compose f g$,它是部分应用的$Compose$对两个函子。

    在$fmap$的实现中,我们对数据构造子$Compose$进行模式匹配。它的参数$gfa$是类型$g (f a)$。我们使用一个$fmap$来“获取$g$下的内容”。然后我们使用$ (fmap h) $来获取$f$下的内容。编译器通过分析类型来知道应该使用哪个$fmap$。

    你可以将一个复合函子视为一个容器中的容器。例如,$ [] $与$Maybe$的组合是一个可选值的列表。

    \begin{exercise}
        定义一个$Functor$之后$Contravariant$的组合。提示:你可以重用$Compose$,但你必须提供一个不同的实例声明。
    \end{exercise}

    \subsection{Category of Categories\\范畴的范畴}

    我们可以将函子视为范畴之间的箭头。正如我们刚刚看到的,函子是可组合的,检查这个组合是结合律的很容易。我们还为每个范畴提供了一个恒等(自)函子。因此,范畴本身似乎形成了一个范畴,我们称之为$\mathbf{Cat}$。

    这是数学家开始担心“大小”问题的地方。这是指代存在悖论的简写。因此,正确的表达是$\mathbf{Cat}$是一个小范畴的范畴。但是只要我们不涉及存在性的证明,我们就可以忽略大小问题。

\end{document}

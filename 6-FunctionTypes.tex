\documentclass[DaoFP]{subfiles}
\begin{document}
 \setcounter{chapter}{5}

 \chapter{Function Types\\函数类型}

 在函数式编程中,还有一种核心的组合方式。当你将一个函数作为参数传递给另一个函数时,外部函数可以将这个参数作为自身机制中的一个可插拔部分。这使得你可以实现一个通用的排序算法,该算法接受任意的比较函数。

 如果我们将函数建模为对象之间的箭头,那么拥有一个函数作为参数意味着什么呢?

 我们需要一种将函数对象化的方法,以便定义以“箭头对象”作为源或目标的箭头。接受函数作为参数或返回函数的函数被称为\emph{higher-order function}(高阶函数)。高阶函数是函数式编程的工作马。

 \subsection{Elimination rule\\消去规则}

 函数的定义性特征是它可以应用于一个参数来产生结果。我们已经根据组合定义了函数应用:

 \[
  \begin{tikzcd}
   1
   \arrow[d, "x"']
   \arrow[rd, "y"]
   \\
   a
   \arrow[r, "f"']
   & b
  \end{tikzcd}
 \]
 这里,$f$被表示为从$a$到$b$的箭头,但我们希望能够将$f$替换为箭头对象的一个元素,或者用数学家的话来说,替换为指数对象$b^a$;在编程中,我们称之为函数类型\hask{a->b}。

 给定$b^a$的一个元素和$a$的一个元素,函数应用应该产生$b$的一个元素。换句话说,给定一对元素:
 \begin{align*}
  f &\colon 1 \to b^a \\
  x &\colon 1 \to a
 \end{align*}
 它应该产生一个元素:
 \[y \colon 1 \to b \]

 请记住,这里$f$表示$b^a$的一个元素。之前,它是从$a$到$b$的一个箭头。

 我们知道,一对元素$(f, x)$等价于积$b^a \times a$的一个元素。因此,我们可以将函数应用定义为一个单一的箭头:
 \[\varepsilon_{a b} \colon b^a \times a \to b\]
 这样,应用的结果$y$由以下交换图定义:
 \[
  \begin{tikzcd}
   1
   \arrow[d, "{(f, x)}"']
   \arrow[rd, "y"]
   \\
   b^a \times a
   \arrow[r, "\varepsilon_{a b}"']
   & b
  \end{tikzcd}
 \]
 函数应用是函数类型的\emph{消去规则}。

 当有人给你一个函数对象的元素时,你唯一能做的就是使用$\varepsilon$将它应用于参数类型的一个元素。

 \subsection{Introduction rule\\引入规则}
 为了完成函数对象的定义,我们还需要引入规则。

 首先,假设有一种方法可以从其他对象构造函数对象$b^a$。这意味着存在一个箭头
 \[h \colon c \to b^a\]
 我们知道我们可以使用$\varepsilon_{a b}$消去$h$的结果,但我们必须首先将其乘以$a$。因此,首先将$c$乘以$a$,然后使用函子性将其映射到$b^a \times a$。

 函子性让我们可以将一对箭头应用于一个积以得到另一个积。在这里,箭头对是$(h, id_a)$(我们希望将$c$转变为$b^a$,但我们对修改$a$不感兴趣)
 \[ c \times a \xrightarrow{h \times id_a} b^a \times a \]
 现在我们可以用函数应用继续将其映射到$b$:
 \[ c \times a \xrightarrow{h \times id_a} b^a \times a \xrightarrow{\varepsilon_{a b}} b\]
 这个复合箭头定义了一个我们称之为$f$的映射:
 \[f \colon c \times a \to b\]
 以下是相应的图示:

 \[
  \begin{tikzcd}
   c \times a
   \arrow[d, dashed, "h \times id_a"']
   \arrow[rd, "f"]
   \\
   b^a \times a
   \arrow[r, "\varepsilon"']
   & b
  \end{tikzcd}
 \]
 这个交换图告诉我们,给定一个$h$,我们可以构造一个$f$;但我们也可以要求反过来:每一个映射$f \colon c \times a \to b$都应该唯一地定义一个到指数对象的映射$h \colon c \to b^a$。

 我们可以利用这个属性,即箭头集合之间的一一对应关系来定义指数对象。这是函数对象$b^a$的\emph{引入规则}。

 我们已经看到积是通过它的映射进入性质定义的。而函数应用则定义为从一个积中映射出来的\emph{映射出}。

 \subsection{Currying\\柯里化}

 有几种方式可以看待这个定义。一种是将其视为柯里化的一个例子。

 到目前为止,我们只考虑了单参数的函数。这并不是一个真正的限制,因为我们总是可以实现一个双参数的函数作为一个从积到函数的(单参数)函数。函数对象定义中的$f$就是这样一个函数:
 \begin{haskell}
  f :: (c, a) -> b
 \end{haskell}
 而\hask{h}则是一个返回函数的函数:
 \begin{haskell}
  h :: c -> (a -> b)
 \end{haskell}
 柯里化就是这两个箭头集合之间的同构。

 这种同构可以通过一对(高阶)函数在Haskell中表示出来。由于在Haskell中,柯里化适用于任何类型,这些函数使用类型变量编写——它们是\emph{多态}的:
 \begin{haskell}
  curry   :: ((c, a) -> b)   -> (c -> (a -> b))
 \end{haskell}

 \begin{haskell}
  uncurry :: (c -> (a -> b)) -> ((c, a) -> b)
 \end{haskell}
 换句话说,函数对象定义中的$h$可以写成:
 \[ h = curry\, f \]

 当然,以这种方式书写,\hask{curry}和\hask{uncurry}的类型对应于函数对象而非箭头。这种区别通常被忽略,因为指数的\emph{元素}和定义它们的\emph{箭头}之间有一一对应的关系。当我们用终端对象替换任意对象$c$时,这一点很容易看出。我们得到:

 \[
  \begin{tikzcd}
   1 \times a
   \arrow[d, dashed, "h \times id_a"']
   \arrow[rd, "f"]
   \\
   b^a \times a
   \arrow[r, "\varepsilon_{a b}"']
   & b
  \end{tikzcd}
 \]
 在这种情况下,$h$是对象$b^a$的一个元素,$f$是从$1 \times a$到$b$的一个箭头。但我们知道$1 \times a$同构于$a$,所以实际上,$f$是从$a$到$b$的一个箭头。

 因此,从现在开始,我们将\hask{->}箭头称为$\to$箭头,而不必过多纠缠于此。这种现象的正确表达方式是说这个范畴是自封闭的。

 我们可以将$\varepsilon_{a b}$写成Haskell函数\hask{apply}:
 \begin{haskell}
  apply :: (a -> b, a) -> b
  apply (f, x) = f x
 \end{haskell}
 但这只是一个语法技巧:函数应用内置于语言中:\hask{f x}意味着\hask{f}应用于\hask{x}。其他编程语言要求函数的参数用括号括起来,而在Haskell中则不需要。

 尽管将函数应用定义为一个单独的函数似乎是多余的,Haskell库确实提供了一个用于此目的的中缀操作符\hask{$}:
 \begin{haskell}
  ($) :: (a -> b) -> a -> b
f $ x = f x
\end{haskell}
但这个技巧在于常规的函数应用绑定到左边,例如,\hask{f x y}与\hask{(f x) y}相同;但美元符号绑定到右边,因此
\begin{haskell}
f  $ g x
\end{haskell}
与\hask{f (g x)}相同。在第一个例子中,\hask{f}必须是一个(至少)双参数函数;在第二个例子中,它可以是一个单参数函数。

在Haskell中,柯里化无处不在。双参数函数几乎总是被写作返回一个函数的函数。由于函数箭头\hask{->}绑定到右边,因此不需要为此类类型加括号。例如,配对构造函数具有如下签名:
\begin{haskell}
pair :: a -> b -> (a, b)
\end{haskell}
你可以将其视为一个双参数函数,返回一个配对,或者是一个单参数函数,返回一个单参数函数\hask{b->(a, b)}。因此,可以部分应用此类函数,其结果为另一个函数。例如,我们可以定义:
\begin{haskell}
pairWithTen :: a -> (Int, a)
pairWithTen = pair 10 -- partial application of pair
\end{haskell}
\subsection{Relation to lambda calculus\\与λ演算的关系}

另一种看待函数对象定义的方法是将$c$解释为$f$定义的环境类型。在这种情况下,通常将环境称为$\Gamma$。箭头被解释为使用$\Gamma$中定义的变量的表达式。

考虑一个简单的例子,表达式:
\[a x^2 + b x + c\]
你可以将其看作是由一个实数三元组$(a, b, c)$和一个变量$x$(假设为一个复数)参数化的。该三元组是一个积$\mathbb{R} \times \mathbb{R} \times \mathbb{R}$的元素。这个积就是我们的表达式的环境$\Gamma$。

变量$x$是$\mathbb{C}$的一个元素。表达式是从积$\Gamma \times \mathbb{C}$到结果类型(这里也是$\mathbb{C}$)的箭头:
\[f \colon \Gamma \times \mathbb{C} \to \mathbb{C} \]
这是从一个积映射出的映射,因此我们可以用它来构造一个函数对象$\mathbb{C}^{\mathbb{C}}$并定义一个映射$h \colon \Gamma \to \mathbb{C}^{\mathbb{C}}$
\[
\begin{tikzcd}
\Gamma \times \mathbb{C}
\arrow[d, dashed, "h \times id_{\mathbb{C}}"']
\arrow[rd, "f"]
\\
\mathbb{C}^{\mathbb{C}} \times \mathbb{C}
\arrow[r, "\varepsilon"']
& \mathbb{C}
\end{tikzcd}
\]
这个新的映射$h$可以被看作是函数对象的构造器。所得的函数对象表示所有从$\mathbb{C}$到$\mathbb{C}$的函数,这些函数可以访问环境$\Gamma$,即参数三元组$(a, b, c)$。

对应于我们最初的表达式$a x^2 + b x + c$,在$\mathbb{C}^{\mathbb{C}}$中有一个特定的函数,我们写作:
\[  \lambda x . \,a x^2 + b x + c \]
或者在Haskell中,使用\index{backslash}反斜杠代替$\lambda$:
\begin{haskell}
\x -> a * x^2 + b * x + c
\end{haskell}

箭头$h \colon \Gamma \to \mathbb{C}^{\mathbb{C}}$由箭头$f$唯一确定。此映射生成我们称之为$\lambda x . f$的函数。

一般而言,函数对象的定义图变为:
\[
\begin{tikzcd}
\Gamma \times a
\arrow[d, dashed, "{h \times id_a}"']
\arrow[rd, "f"]
\\
b^a \times a
\arrow[r, "\varepsilon"']
& b
\end{tikzcd}
\]

提供表达式$f$的自由参数的环境$\Gamma$是代表参数类型的多个对象的积(在我们的例子中是$\mathbb{R} \times \mathbb{R} \times \mathbb{R}$)。

空环境由终端对象$1$表示,积的单位。在这种情况下,$f$只是一个从$a$到$b$的箭头,而$h$只是从函数对象$b^a$中挑选出与$f$对应的一个元素。

需要牢记的是,通常情况下,函数对象表示依赖于外部参数的函数。这样的函数被称为\index{closure}\emph{闭包}。闭包是从其环境中捕获值的函数。

以下是我们例子在Haskell中的翻译。与$f$对应的是一个表达式:
\begin{haskell}
(a :+ 0) * x * x + (b :+ 0) * x + (c :+ 0)
\end{haskell}
如果我们用\hask{Double}来近似$\mathbb{R}$,我们的环境是一个积\hask{(Double, Double, Double)}。类型\hask{Complex}由另一个类型参数化——这里我们再次使用\hask{Double}:
\begin{haskell}
type C = Complex Double
\end{haskell}
从\hask{Double}到\hask{C}的转换是通过将虚部设为零来完成的,例如\hask{(a :+ 0)}。

相应的箭头$h$接受环境并生成一个类型为\hask{C -> C}的闭包:
\begin{haskell}
h :: (Double, Double, Double) -> (C -> C)
h (a, b, c) = \x -> (a :+ 0) * x * x + (b :+ 0) * x + (c :+ 0)
\end{haskell}

\subsection{Modus ponens\\演绎规则}

在逻辑中,函数对象对应于一个蕴涵。从终端对象到函数对象的箭头是该蕴涵的证明。函数应用$\varepsilon$对应于逻辑学家称之为\emph{演绎规则}的概念:如果你有$A \Rightarrow B$的证明以及$A$的证明,那么这就构成了$B$的证明。

\section{Sum and Product Revisited\\再谈和与积}

当函数与其他类型的元素获得相同的地位时,我们拥有了直接将图表转换为代码的工具。

\subsection{Sum types\\和类型}

我们从和类型的定义开始。
\[
\begin{tikzcd}
a
\arrow[dr,  bend left, "\text{Left}"']
\arrow[ddr, bend right, "f"']
&& b
\arrow[dl, bend right, "\text{Right}"]
\arrow[ddl, bend left, "g"]
\\
&a + b
\arrow[d, dashed, "h"]
\\
& c
\end{tikzcd}
\]
我们说过,箭头对$(f, g)$唯一地决定了从和类型到其他对象的映射$h$。我们可以使用一个高阶函数简洁地将其写成:
\begin{haskell}
h = mapOut (f, g)
\end{haskell}
其中:
\begin{haskell}
mapOut :: (a -> c, b -> c) -> (Either a b -> c)
mapOut (f, g) = \aorb -> case aorb of
Left  a -> f a
Right b -> g b
\end{haskell}
这个函数以一对函数作为参数,并返回一个函数。

首先,我们通过模式匹配提取\hask{(f, g)}中的\hask{f}和\hask{g}。然后我们使用lambda构造一个新函数。这个lambda接受一个类型为\hask{Either a b}的参数,我们称之为\hask{aorb},并对其进行case分析。如果它是用\hask{Left}构造的,我们将其内容传递给\hask{f},否则传递给\hask{g}。

请注意,我们返回的函数是一个闭包。它从环境中捕获\hask{f}和\hask{g}。

我们实现的函数与图示非常相似,但它并不是通常的Haskell风格。Haskell程序员更喜欢将多参数函数柯里化。此外,如果可能的话,他们更喜欢消除lambdas。

以下是相同函数的Haskell标准库中的版本,它以小写字母\hask{either}命名:
\begin{haskell}
either :: (a -> c) -> (b -> c) -> Either a b -> c
either f _ (Left x)     =  f x
either _ g (Right y)    =  g y
\end{haskell}
另一方向的双射(从$h$到对$(f, g)$)也遵循图的箭头:

\begin{haskell}
unEither :: (Either a b -> c) -> (a -> c, b -> c)
unEither h = (h . Left, h . Right)
\end{haskell}

\subsection{Product types\\积类型}

积类型通过它们的映射入属性对偶地定义。
\[
\begin{tikzcd}
& c
\arrow[d, dashed, "h"]
\arrow[ddl, bend right, "f"']
\arrow[ddr, bend left, "g"]
\\
&a \times b
\arrow[dl,  "\text{fst}"]
\arrow[dr,   "\text{snd}"']
\\
a && b
\end{tikzcd}
\]
这是直接从这个图在Haskell中的读取:
\begin{haskell}
h :: (c -> a, c -> b) -> (c -> (a, b))
h (f, g) = \c -> (f c, g c)
\end{haskell}
这是用Haskell风格编写的版本,作为中缀操作符\hask{&&&}:
\begin{haskell}
(&&&) :: (c -> a) -> (c -> b) -> (c -> (a, b))
(f &&& g) c = (f c, g c)
\end{haskell}
双射的另一个方向是:
\begin{haskell}
fork :: (c -> (a, b)) -> (c -> a, c -> b)
fork h = (fst . h, snd . h)
\end{haskell}
这也与图的读取密切相关。

\subsection{Functoriality revisited\\函子性再探}

无论是和类型还是积类型都是函子的,这意味着我们可以将函数应用到它们的内容上。我们准备好将这些图转化为代码。

这是和类型的函子性:

\[
\begin{tikzcd}
a
\arrow[d, "f"]
\arrow[dr,  bend left, "\text{Left}"']
&& b
\arrow[d, "g"]
\arrow[dl, bend right, "\text{Right}"]
\\
a'
\arrow[rd, "\text{Left}"']
&a + b
\arrow[d, dashed, "h"]
& b'
\arrow[ld, "\text{Right}"]
\\
& a' + b'
\end{tikzcd}
\]
读取这个图,我们可以立即使用\hask{either}编写$h$:
\begin{haskell}
h f g = either (Left . f) (Right . g)
\end{haskell}
或者我们可以将其扩展并称之为\hask{bimap}:
\begin{haskell}
bimap :: (a -> a') -> (b -> b') -> Either a b -> Either a' b'
bimap f g (Left  a) = Left  (f a)
bimap f g (Right b) = Right (g b)
\end{haskell}
对于积类型也类似:
\[
\begin{tikzcd}
& a \times b
\arrow[d, dashed, "h"]
\arrow[dl,  "\text{fst}"']
\arrow[dr,   "\text{snd}"]
\\
a
\arrow[d, "f"']
&a' \times b'
\arrow[dl,  "\text{fst}"]
\arrow[dr,   "\text{snd}"']
& b
\arrow[d, "g"]
\\
a' && b'
\end{tikzcd}
\]
$h$可以写成:
\begin{haskell}
h f g = (f . fst) &&& (g . snd)
\end{haskell}
或者可以将其扩展为:
\begin{haskell}
bimap :: (a -> a') -> (b -> b') -> (a, b) -> (a', b')
bimap f g (a, b) = (f a, g b)
\end{haskell}
在这两种情况下,我们都称这个高阶函数为\hask{bimap},因为在Haskell中,和类型和积类型都是一个更通用类\hask{Bifunctor}的实例。

\section{Functoriality of the Function Type\\函数类型的函子性}

函数类型或指数类型也是函子的,但有一个不同之处。我们感兴趣的是从$b^a$到$b'^{a'}$的映射,其中带撇号的对象通过一些箭头与不带撇号的对象相关联——这些箭头有待确定。

指数类型通过其映射入属性定义,因此如果我们在寻找
\[k \colon b^a \to b'^{a'} \]
我们应该画出有$k$作为映射入$b'^{a'}$的图。我们通过将$b^a$替换为$c$并将带撇号的对象替换为不带撇号的对象,从原始定义中得到这个图:

\[
\begin{tikzcd}
b^a \times a'
\arrow[d, dashed, "k \times id_a"']
\arrow[rd, "g"]
\\
b'^{a'} \times a'
\arrow[r, "\varepsilon"']
& b'
\end{tikzcd}
\]
问题是:我们可以找到一个箭头$g$来完成这个图吗?
\[g \colon b^a \times a' \to b'\]
如果我们找到这样一个$g$,它将唯一地确定我们的$k$。

思考这个问题的方法是考虑如何实现$g$。它将$b^a \times a'$的积作为其参数。可以将其视为一个对:从$a$到$b$的函数对象的一个元素和一个$a'$的元素。我们唯一能对函数对象做的事情是将其应用于某些东西。但$b^a$需要一个类型为$a$的参数,而我们所拥有的只是$a'$。除非有人给我们一个箭头$a' \to a$,否则我们无能为力。该箭头应用于$a'$将生成$b^a$的参数。然而,应用的结果是$b$类型,而$g$应该生成一个$b'$。再一次,我们需要一个箭头$b \to b'$来完成我们的赋值。

这可能听起来很复杂,但底线是我们需要两个箭头来连接带撇号和不带撇号的对象。不同之处在于第一个箭头从$a'$到$a$,这在通常的函子性考虑中似乎是逆向的。为了将$b^a$映射到$b'^{a'}$,我们需要一对箭头:
\begin{align*}
f &\colon a' \to a \\
g &\colon b \to b'
\end{align*}

这在Haskell中更容易解释。我们的目标是实现一个函数\hask{a' -> b'},给定一个函数\hask{h :: a -> b}。

这个新函数接受一个类型为\hask{a'}的参数,因此在我们将其传递给\hask{h}之前,我们需要将\hask{a'}转换为\hask{a}。这就是为什么我们需要一个函数\hask{f :: a' -> a}。

由于\hask{h}生成一个\hask{b},而我们希望返回一个\hask{b'},我们需要另一个函数\hask{g :: b -> b'}。所有这些都很好地适合于一个高阶函数:
\begin{haskell}
dimap :: (a' -> a) -> (b -> b') -> (a -> b) -> (a' -> b')
dimap f g h = g . h . f
\end{haskell}
类似于\hask{bimap}是类型类\hask{Bifunctor}的一个接口,\hask{dimap}是类型类\hask{Profunctor}的成员。

\section{Bicartesian Closed Categories\\双笛卡尔封闭范畴}

一个同时为任意对象对定义了积和指数,并且具有终端对象的范畴被称为\emph{笛卡尔封闭}。这个想法是,同态集对于所讨论的范畴来说并不是外来的东西:这个范畴在形成同态集的操作下是“封闭的”。

如果范畴还具有和类型(余积)和初始对象,那么它被称为\emph{双笛卡尔封闭}。

这是建模编程语言的最小结构。

使用这些操作构造的数据类型被称为\emph{代数数据类型}。我们有加法、乘法和指数(但没有减法或除法)类型;以及我们在高中代数中熟悉的所有定律。它们在同构意义上得到满足。还有一个代数定律我们还没有讨论。

\subsection{Distributivity\\分配性}

数的乘法对加法是分配的。我们应该期望在双笛卡尔封闭范畴中也是如此吗?
\[b \times a + c \times a \cong (b + c) \times a\]

从左到右的映射很容易构造,因为它同时是一个和类型的映射出和积的映射入。我们可以通过逐步将其分解为更简单的映射来构造它。在Haskell中,这意味着实现一个函数:
\begin{haskell}
dist :: Either (b, a) (c, a) -> (Either b c, a)
\end{haskell}
一个从左边的和类型映射出的映射由一对箭头给出:
\begin{align*}
f &\colon b\times a \to (b + c) \times a \\
g &\colon c\times a \to (b + c) \times a
\end{align*}
\[
\begin{tikzcd}
b \times a
\arrow[dr,  bend left, "\text{Left}"]
\arrow[ddr, bend right, "f"']
&& c \times a
\arrow[dl, bend right, "\text{Right}"']
\arrow[ddl, bend left, "g"]
\\
& b \times a + c \times a
\arrow[d, dashed, "\text{dist}"]
\\
& (b + c) \times a
\end{tikzcd}
\]

我们将其写成Haskell代码:
\begin{haskell}
dist = either f g
where
f   :: (b, a) -> (Either b c, a)
g   :: (c, a) -> (Either b c, a)
\end{haskell}
在\index{\hask{where}} \hask{where}子句中定义辅助函数的实现。

现在我们需要实现$f$和$g$。它们是积中的映射,因此每一个都对应于一对箭头。例如,第一个由一对箭头给出:
\begin{align*}
f' &\colon b \times a \to (b + c) \\
f'' &\colon b \times a \to a
\end{align*}
\[
\begin{tikzcd}
& b \times a
\arrow[d, dashed, "f"]
\arrow[ddl, bend right, "f'"']
\arrow[ddr, bend left, "f''"]
\\
& (b + c) \times a
\arrow[dl,  "\text{fst}"]
\arrow[dr,   "\text{snd}"']
\\
b + c && a
\end{tikzcd}
\]

在Haskell中:
\begin{haskell}
f = f' &&& f''
f'  :: (b, a) -> Either b c
f'' :: (b, a) -> a
\end{haskell}
第一个箭头可以通过投影第一个分量$b$并使用$\text{Left}$来构造和类型。第二个只是投影$\text{snd}$:
\begin{align*}
f' &= \text{Left} \circ \text{fst} \\
f'' &= \text{snd}
\end{align*}
类似地,我们将$g$分解为一对$g'$和$g''$:
\[
\begin{tikzcd}
& c \times a
\arrow[d, dashed, "g"]
\arrow[ddl, bend right, "g'"']
\arrow[ddr, bend left, "g''"]
\\
& b + c \times a
\arrow[dl,  "\text{fst}"]
\arrow[dr,   "\text{snd}"']
\\
(b + c) && a
\end{tikzcd}
\]

将所有这些组合在一起,我们得到:
\begin{haskell}
dist = either f g
where
f   = f' &&& f''
f'  = Left . fst
f'' = snd
g   = g' &&& g''
g'  = Right . fst
g'' = snd
\end{haskell}
这些是辅助函数的类型签名:
\begin{haskell}
f   :: (b, a) -> (Either b c, a)
g   :: (c, a) -> (Either b c, a)
f'  :: (b, a) -> Either b c
f'' :: (b, a) -> a
g'  :: (c, a) -> Either b c
g'' :: (c, a) -> a
\end{haskell}
它们也可以内联以生成这种简洁形式:
\begin{haskell}
dist = either ((Left . fst) &&& snd) ((Right . fst) &&& snd)
\end{haskell}

这种编程风格被称为\emph{无点式},因为它省略了参数(点)。出于可读性原因,Haskell程序员更喜欢更明确的风格。上面的函数通常会实现为:
\begin{haskell}
dist (Left  (b, a)) = (Left  b, a)
dist (Right (c, a)) = (Right c, a)
\end{haskell}

请注意,我们只使用了和类型和积的定义。双射的另一个方向需要使用指数,因此它仅在双笛卡尔\emph{封闭}范畴中有效。这在Haskell的直接实现中并不立即显现:
\begin{haskell}
undist :: (Either b c, a) -> Either (b, a) (c, a)
undist (Left b, a)  = Left (b, a)
undist (Right c, a) = Right (c, a)
\end{haskell}
但这是因为在Haskell中柯里化是隐式的。

这是这个函数的无点式版本:
\begin{haskell}
undist = uncurry (either (curry Left) (curry Right))
\end{haskell}
这可能不是最易读的实现,但它强调了我们需要指数的事实:我们使用了\hask{curry}和\hask{uncurry}来实现这个映射。

我们将在后续章节中使用更强大的工具:伴随来回到这个恒等式。

\begin{exercise}
证明:
\[ 2 \times a \cong a + a \]
其中$2$是布尔类型。首先用图形方式证明,然后实现两个Haskell函数来见证这个同构。
\end{exercise}
\end{document}
\documentclass[11pt, book]{memoir}
\usepackage[UTF8]{ctex}
\settrims{0pt}{0pt} % page and stock same size
\settypeblocksize{*}{34.5pc}{*} % {height}{width}{ratio}
\setlrmargins{*}{*}{1} % {spine}{edge}{ratio}
\setulmarginsandblock{1in}{1in}{*} % height of typeblock computed
\setheadfoot{\onelineskip}{2\onelineskip} % {headheight}{footskip}
\setheaderspaces{*}{1.5\onelineskip}{*} % {headdrop}{headsep}{ratio}
\checkandfixthelayout

\chapterstyle{bianchi}
\newcommand{\titlefont}{\normalfont\Huge\bfseries}
\renewcommand{\chaptitlefont}{\titlefont}

\usepackage{subfiles}
\usepackage{makeidx}

\usepackage{amsfonts}
\usepackage{amssymb}
\usepackage{amsthm}
\usepackage{amsmath}
\usepackage{tikz-cd}
\usetikzlibrary{backgrounds}
\usepackage{float}
\usepackage{xcolor}
\usepackage{stix}

\newtheorem{exercise}{Exercise}[section]
\newcommand{\exinline}[1]{(\refstepcounter{exercise}Exercise~\theexercise\label{#1})}

\usepackage{minted}
\usepackage{etoolbox}

\usepackage[bookmarksnumbered]{hyperref}
\hypersetup{
  colorlinks,
  linkcolor={red!40!black},
  citecolor={red!40!black},
  urlcolor={blue!40!black}
}

\newcommand{\cat}[1]{\mathcal{#1}}%a generic category
\newcommand{\Cat}[1]{\mathbf{#1}}%a named category
\newcommand{\Set}{\Cat{Set}}

\DeclareRobustCommand{\hask}[1]{\ifmmode\text{\texttt{#1}}\else\mintinline{Haskell}{#1}\fi}
%\newcommand{\hask}[1]{\mintinline{Haskell}{#1}}
\newenvironment{haskell}
{\VerbatimEnvironment
\fvset{formatcom=\upshape} % Use upright shape for this environment
  \begin{minted}[escapeinside=??, mathescape=true, tabsize=1, bgcolor=black!3, xleftmargin=3pt ]{Haskell}}
  {\end{minted}}
    \makeatletter
% replace \medskip before and after the box with nothing, i.e., remove it
    \patchcmd{\minted@colorbg}{\medskip}{}{}{}
\patchcmd{\endminted@colorbg}{\medskip}{}{}{}
\makeatother

% lens brackets for anamorphisms
\newcommand{\llens}{\mathopen{[\!(}}
\newcommand{\rlens}{\mathclose{)\!]}}

\makeindex

\begin{document}
  \setcounter{tocdepth}{4}
  \setcounter{secnumdepth}{4}
  \frontmatter

  \title{\huge The Dao of Functional Programming}
  \author{\Large Bartosz Milewski }

  \date{\vfill (Last updated: \today)}

  \maketitle

  \tableofcontents*

  \clearpage

  \section{序言}

  大多数编程教材,遵循 Brian Kernighan 的做法,都是以``Hello World!''开始的。这很自然,因为我们想立即通过让计算机执行命令并打印出这些著名的词语来获得满足感。但计算机编程的真正掌握远比这更深刻,急于求成可能只会给你一种虚假的力量感,而实际上你只是在模仿大师。如果你的目标只是学习一项有用且薪水丰厚的技能,那么,尽管去写你的``Hello World!''程序吧。有大量的书籍和课程可以教你如何用你选择的任何语言编写代码。然而,如果你真的想深入编程的本质,你需要耐心和坚持。

  范畴论是数学的一个分支,它提供了与编程实践经验相符的抽象。借用克劳塞维茨的一句话:编程只是以其他方式继续数学的过程。当用数据类型和函数来解释范畴论的许多复杂概念时,程序员们可能比专业数学家更容易理解它们。

  当我面对新的范畴概念时,我常常会在 Wikipedia 或 nLab 上查找它们,或者重新阅读 Mac Lane 或 Kelly 的某一章。这些都是很好的资源,但它们要求你对主题有一定的熟悉度,并且有能力填补其中的空白。本书的一个目标是提供必要的引导,以继续学习范畴论。

  在范畴论和计算机科学中,有许多民间知识在文献中是找不到的。当你通过干巴巴的定义和定理来学习时,很难获得有用的直觉。我尽可能地提供了那些缺失的直觉,并解释了不仅是什么,还解释了为什么。

  本书的标题暗示了 Benjamin Hoff 的``The Tao of Pooh''和 Robert Pirsig 的``Zen and the Art of Motorcycle Maintenance'',这两本书都是西方人尝试吸收东方哲学元素的作品。粗略地说,范畴论之于编程,正如道(Dao)\footnote{Dao 是 Tao 的现代拼写}之于西方哲学。许多范畴论的定义在第一次阅读时显得毫无意义,但随着时间的推移,你会逐渐理解其中深刻的智慧。如果要用一句话总结范畴论,那就是:``事物通过它们与宇宙的关系来定义。''

  \subsection{集合论}

  传统上,集合论被认为是数学的基础,尽管最近类型论也在争夺这一头衔。从某种意义上说,集合论是数学的汇编语言,包含了许多实现细节,这些细节往往会掩盖高级思想的表达。

  范畴论并不是要取代集合论,通常它被用来构建抽象,然后用集合来建模。事实上,范畴论的基本定理尤内达引理将范畴与集合论中的模型联系起来。我们可以在计算机图形学中找到有用的直觉,在那里我们构建和操作抽象世界,直到最后一刻才将它们投影并采样到数字显示器上。

  为了学习范畴论,不必精通集合论。但一些基本知识是必要的。例如,集合包含元素的概念。我们说,给定一个集合 $S$ 和一个元素 $a$,可以问 $a$ 是否属于 $S$。这个陈述写作 $a \in S$ ($a$ 是 $S$ 的成员)。也可以有一个不包含任何元素的空集合。

  集合元素的重要属性是它们可以比较是否相等。给定两个元素 $a \in S$ 和 $b \in S$,我们可以问:$a$ 是否等于 $b$?或者我们可以强加条件 $a = b$,如果 $a$ 和 $b$ 是通过不同的方法从集合 $S$ 中选择的元素的结果。集合元素的相等性是所有范畴论中交换图的本质。

  两个集合的笛卡尔积 $S \times T$ 被定义为一组所有的元素对 $\langle s, t \rangle$,其中 $s \in S$ 和 $t \in T$。

  一个从源集合(称为 $f$ 的\emph{定义域})到目标集合(称为 $f$ 的\emph{值域})的函数 $f \colon S \to T$ 也被定义为一组对。这些对的形式为 $\langle s, t \rangle$,其中 $t = f s$。这里 $f s$ 是函数 $f$ 作用在参数 $s$ 上的结果。你可能更熟悉函数应用的符号 $f(s)$,但这里我将遵循 Haskell 的惯例,省略括号(以及多个变量的逗号)。

  在编程中,我们习惯于通过一系列指令来定义函数。我们提供一个参数 $s$ 并应用指令,最终产生结果 $t$。我们经常担心计算结果可能需要多长时间,或者算法是否会终止。在数学中,我们假设对于任何给定的参数 $s \in S$,结果 $t \in T$ 是立即可得的,并且是唯一的。在编程中,我们称这种函数为纯函数且是全函数。

  \subsection{约定}

  我尽量使全书的符号保持一致。它大致基于 nLab 中流行的风格。

  特别地,我决定使用小写字母,如 $a$ 或 $b$ 来表示范畴中的对象,并使用大写字母如 $S$ 来表示集合(尽管集合是集合与函数范畴中的对象)。通用范畴有名称如 $\cat C$ 或 $\cat D$,而特定的范畴有名称如 $\Set$ 或 $\Cat{Cat}$。

  编程示例以 Haskell 语言编写。尽管这不是一本 Haskell 手册,但语言构造的引入是逐渐进行的,以帮助读者理解代码。Haskell 语法通常基于数学符号,这是一个额外的优势。程序片段以如下格式编写:
  \begin{haskell}
    apply :: (a -> b, a) -> b
    apply (f, x) = f x
  \end{haskell}

  \mainmatter

  \subfile{1-CleanSlate}
  \subfile{2-Composition}
  \subfile{3-Isomorphism}
  \subfile{4-SumTypes}
  \subfile{5-ProductTypes}
  \subfile{6-FunctionTypes}
  \subfile{7-Recursion}
  \subfile{8-Functors}
  \subfile{9-NaturalTransformations}
  \subfile{10-Adjunctions}
  \subfile{11-DependentTypes}
  \subfile{12-Algebras}
  \subfile{13-Coalgebras}
  \subfile{14-Monads}
  \subfile{15-MonadsAdjunctions}
  \subfile{16-Comonads}
  \subfile{17-Ends}
  \subfile{18-Tambara}
  \subfile{19-Kan}
  \subfile{20-Enrichment}

\printindex

\end{document}
